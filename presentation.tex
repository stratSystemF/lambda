\documentclass{beamer}
 
 \usepackage[utf8]{inputenc} 
 \usepackage[T1]{fontenc}
 \usepackage{lmodern}
 \usepackage{graphicx}
 \usepackage[french]{babel}
  
  \usetheme{Warsaw}
   
   \begin{document}

	\title{Immersion dans Coq : un début de formalisation du système F}
	\author{Théo Zimmermann, Thomas Bourgeat}
\frame{\titlepage}

\section{Résumé}

\begin{frame}
Objectif du projet
\begin{itemize}
\item Apprendre le langage Coq.
\pause
\item Apprendre à formaliser un problème en Coq.
\pause
\item {\scriptsize Prouver des résultats simples sur le système F (lemmes de
préservation, d'affaiblissement ...)}
\item {\tiny Prouver la normalisation du système F}
\end{itemize}


\end{frame}

    \begin{frame}
    \frametitle{Qu'avons-nous faits?}
	\begin{itemize}
	\item Définitions (partie 1.2)
	\begin{itemize}
	\item Complétude de l'inférence de kind mais pas de l'inférence
	de types. (Parce que non demandé et qu'on bloquait dessus)
	\end{itemize}
	\pause
	\item Partie 1.3.2 incomplète
	\begin{itemize}
	\item Remplir ça
	\end{itemize}
	\end{itemize}
	Le projet final fait environ 1300 lignes.
    \end{frame}
     
		
\section{Choix de formalisation}


\begin{frame}{De bruijn}
\begin{itemize}
\item De bruijn parce que Vouillon.
\pause
\item On a gardé la définition sans prédicats inductifs de la bonne
formation
\pause
\begin{itemize}
\item on a regretté ce choix.
\end{itemize}
\end{itemize}
\end{frame}

\begin{frame}[fragile]
\begin{itemize}
\item Réduction simultanée des deux côté d'une application.
\pause
\end{itemize}
\begin{verbatim}
reduction t t' ->
reduction s s' ->
reduction (Top.app t s) (Top.app t's').
\end{verbatim}
\pause
\begin{itemize}
\item Nous nous sommes basés sur \emph{A tutorial implementation of a
dependently typed lambda-calculus} (Löh et al.) pour la neutralité et la
normalité. On a étendu ça "naturellement".
\end{itemize}
\end{frame}

\begin{frame}[fragile]
\begin{verbatim}
Inductive normal : term -> Prop :=
| directlyNeutral: forall t, neutral t -> normal t
| absNorm : forall v phi, normal v 
		-> normal (abs phi v)
| absTNorm : forall v k, normal v 
		-> normal (dept k v)
with neutral : term -> Prop :=
| nV: forall n, neutral (var n)
| nApplt: forall phi t, neutral t 
	  -> neutral (applt t phi)
| nApp: forall t t', neutral t -> normal t' 
	-> neutral (Top.app t t').
\end{verbatim}
\end{frame}

\section{Problèmes et difficultés}

\begin{frame}{La connaissance c'est le pouvoir}
\begin{itemize}
\item On a découvert des mots magiques au fur et à mesure:
\pause
\begin{itemize}
\item specialize
\item generalize
\item lia
\item firstorder
\end{itemize}
\pause
\item avant ça on a fait du \emph{assert} à toutes les sauces.
\end{itemize}
\end{frame}

\begin{frame}{Formalisation}
\begin{itemize}
\item Pour construire un prédicat inductif il vaut mieux faire des essais
papiers à la main \emph{avant} d'essayer de faire des preuves de choses
compliquées utilisant ce prédicat.
\pause
\item Trouver la bonne formulation des lemmes. (on a parfois oublié de mettre une
hypothèse de bonne formation qui nous semblait pas nécessaire mais en
fait si).
\pause
\item Faire les cas de base sur papier, on a écrit beaucoup de lemmes
faux, qu'on a momentanément pensé \emph{clairement justes}.
\pause
\item \emph{Choisir des noms qui n'introduisent pas de biais
psychologique}
\end{itemize}
\end{frame}

\begin{frame}{Prochaine étape}
\begin{itemize}
\item Apprendre Ltac
\item Remercier Jérôme Vouillon d'avoir fait un code qui nous a permis
d'apprendre coq, de se débloquer quand on était trop bloqués. 
\end{itemize}
\end{frame}

\end{document}

